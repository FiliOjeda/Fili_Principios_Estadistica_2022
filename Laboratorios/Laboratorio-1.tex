% Options for packages loaded elsewhere
\PassOptionsToPackage{unicode}{hyperref}
\PassOptionsToPackage{hyphens}{url}
%
\documentclass[
]{article}
\title{Laboratorio-1.R}
\author{Junio}
\date{2022-02-04}

\usepackage{amsmath,amssymb}
\usepackage{lmodern}
\usepackage{iftex}
\ifPDFTeX
  \usepackage[T1]{fontenc}
  \usepackage[utf8]{inputenc}
  \usepackage{textcomp} % provide euro and other symbols
\else % if luatex or xetex
  \usepackage{unicode-math}
  \defaultfontfeatures{Scale=MatchLowercase}
  \defaultfontfeatures[\rmfamily]{Ligatures=TeX,Scale=1}
\fi
% Use upquote if available, for straight quotes in verbatim environments
\IfFileExists{upquote.sty}{\usepackage{upquote}}{}
\IfFileExists{microtype.sty}{% use microtype if available
  \usepackage[]{microtype}
  \UseMicrotypeSet[protrusion]{basicmath} % disable protrusion for tt fonts
}{}
\makeatletter
\@ifundefined{KOMAClassName}{% if non-KOMA class
  \IfFileExists{parskip.sty}{%
    \usepackage{parskip}
  }{% else
    \setlength{\parindent}{0pt}
    \setlength{\parskip}{6pt plus 2pt minus 1pt}}
}{% if KOMA class
  \KOMAoptions{parskip=half}}
\makeatother
\usepackage{xcolor}
\IfFileExists{xurl.sty}{\usepackage{xurl}}{} % add URL line breaks if available
\IfFileExists{bookmark.sty}{\usepackage{bookmark}}{\usepackage{hyperref}}
\hypersetup{
  pdftitle={Laboratorio-1.R},
  pdfauthor={Junio},
  hidelinks,
  pdfcreator={LaTeX via pandoc}}
\urlstyle{same} % disable monospaced font for URLs
\usepackage[margin=1in]{geometry}
\usepackage{color}
\usepackage{fancyvrb}
\newcommand{\VerbBar}{|}
\newcommand{\VERB}{\Verb[commandchars=\\\{\}]}
\DefineVerbatimEnvironment{Highlighting}{Verbatim}{commandchars=\\\{\}}
% Add ',fontsize=\small' for more characters per line
\usepackage{framed}
\definecolor{shadecolor}{RGB}{248,248,248}
\newenvironment{Shaded}{\begin{snugshade}}{\end{snugshade}}
\newcommand{\AlertTok}[1]{\textcolor[rgb]{0.94,0.16,0.16}{#1}}
\newcommand{\AnnotationTok}[1]{\textcolor[rgb]{0.56,0.35,0.01}{\textbf{\textit{#1}}}}
\newcommand{\AttributeTok}[1]{\textcolor[rgb]{0.77,0.63,0.00}{#1}}
\newcommand{\BaseNTok}[1]{\textcolor[rgb]{0.00,0.00,0.81}{#1}}
\newcommand{\BuiltInTok}[1]{#1}
\newcommand{\CharTok}[1]{\textcolor[rgb]{0.31,0.60,0.02}{#1}}
\newcommand{\CommentTok}[1]{\textcolor[rgb]{0.56,0.35,0.01}{\textit{#1}}}
\newcommand{\CommentVarTok}[1]{\textcolor[rgb]{0.56,0.35,0.01}{\textbf{\textit{#1}}}}
\newcommand{\ConstantTok}[1]{\textcolor[rgb]{0.00,0.00,0.00}{#1}}
\newcommand{\ControlFlowTok}[1]{\textcolor[rgb]{0.13,0.29,0.53}{\textbf{#1}}}
\newcommand{\DataTypeTok}[1]{\textcolor[rgb]{0.13,0.29,0.53}{#1}}
\newcommand{\DecValTok}[1]{\textcolor[rgb]{0.00,0.00,0.81}{#1}}
\newcommand{\DocumentationTok}[1]{\textcolor[rgb]{0.56,0.35,0.01}{\textbf{\textit{#1}}}}
\newcommand{\ErrorTok}[1]{\textcolor[rgb]{0.64,0.00,0.00}{\textbf{#1}}}
\newcommand{\ExtensionTok}[1]{#1}
\newcommand{\FloatTok}[1]{\textcolor[rgb]{0.00,0.00,0.81}{#1}}
\newcommand{\FunctionTok}[1]{\textcolor[rgb]{0.00,0.00,0.00}{#1}}
\newcommand{\ImportTok}[1]{#1}
\newcommand{\InformationTok}[1]{\textcolor[rgb]{0.56,0.35,0.01}{\textbf{\textit{#1}}}}
\newcommand{\KeywordTok}[1]{\textcolor[rgb]{0.13,0.29,0.53}{\textbf{#1}}}
\newcommand{\NormalTok}[1]{#1}
\newcommand{\OperatorTok}[1]{\textcolor[rgb]{0.81,0.36,0.00}{\textbf{#1}}}
\newcommand{\OtherTok}[1]{\textcolor[rgb]{0.56,0.35,0.01}{#1}}
\newcommand{\PreprocessorTok}[1]{\textcolor[rgb]{0.56,0.35,0.01}{\textit{#1}}}
\newcommand{\RegionMarkerTok}[1]{#1}
\newcommand{\SpecialCharTok}[1]{\textcolor[rgb]{0.00,0.00,0.00}{#1}}
\newcommand{\SpecialStringTok}[1]{\textcolor[rgb]{0.31,0.60,0.02}{#1}}
\newcommand{\StringTok}[1]{\textcolor[rgb]{0.31,0.60,0.02}{#1}}
\newcommand{\VariableTok}[1]{\textcolor[rgb]{0.00,0.00,0.00}{#1}}
\newcommand{\VerbatimStringTok}[1]{\textcolor[rgb]{0.31,0.60,0.02}{#1}}
\newcommand{\WarningTok}[1]{\textcolor[rgb]{0.56,0.35,0.01}{\textbf{\textit{#1}}}}
\usepackage{graphicx}
\makeatletter
\def\maxwidth{\ifdim\Gin@nat@width>\linewidth\linewidth\else\Gin@nat@width\fi}
\def\maxheight{\ifdim\Gin@nat@height>\textheight\textheight\else\Gin@nat@height\fi}
\makeatother
% Scale images if necessary, so that they will not overflow the page
% margins by default, and it is still possible to overwrite the defaults
% using explicit options in \includegraphics[width, height, ...]{}
\setkeys{Gin}{width=\maxwidth,height=\maxheight,keepaspectratio}
% Set default figure placement to htbp
\makeatletter
\def\fps@figure{htbp}
\makeatother
\setlength{\emergencystretch}{3em} % prevent overfull lines
\providecommand{\tightlist}{%
  \setlength{\itemsep}{0pt}\setlength{\parskip}{0pt}}
\setcounter{secnumdepth}{-\maxdimen} % remove section numbering
\ifLuaTeX
  \usepackage{selnolig}  % disable illegal ligatures
\fi

\begin{document}
\maketitle

\begin{Shaded}
\begin{Highlighting}[]
\CommentTok{\# Laboratorio 1 }
\CommentTok{\# Filiberto Lozoya Ojeda}
\CommentTok{\#02/02/2022}


\CommentTok{\# PARTE II VARIABLES {-}{-}{-}{-}{-}{-}{-}{-}{-}{-}{-}{-}{-}{-}{-}{-}{-}{-}{-}{-}{-}{-}{-}{-}{-}{-}{-}{-}{-}{-}{-}{-}{-}{-}{-}{-}{-}{-}{-}{-}{-}{-}{-}{-}{-}{-}{-}{-}{-}{-}{-}{-}{-}{-}}

\CommentTok{\#Problema 1:}

\CommentTok{\#Identifique el tipo de variable (cualitativa o cuantitativa) para la lista de preguntas de una encuesta}
\CommentTok{\#aplicada a estudiantes universitarios en una clase de estadística:}
  
\CommentTok{\#Nombre de estudiante.}
\CommentTok{\#Cualitativa}
\CommentTok{\# Fecha de nacimiento (p. Ej., 21/10/1995).}
\CommentTok{\#Cuantitativa}
\CommentTok{\#Edad (en años).}
\CommentTok{\#Cuantitativa}
\CommentTok{\#Dirección de casa (por ejemplo, 1234 Ave. Alamo).}
\CommentTok{\#Cualitativa}
\CommentTok{\#Número de teléfono (por ejemplo, 510{-}123{-}4567)}
\CommentTok{\#Cuantitativas}
\CommentTok{\#Área principal de estudio.}
\CommentTok{\#Cualitativas}
\CommentTok{\#Grado de año universitario: primer año, segundo año, tercer año, último año.}
\CommentTok{\#Cualitativa}
\CommentTok{\#Puntaje en la prueba de mitad de período (basado en 100 puntos posibles).}
\CommentTok{\#cuantitativa}
\CommentTok{\#Calificación general: A, B, C, D, F.}
\CommentTok{\#cualitativa}
\CommentTok{\#Tiempo (en minutos) para completar la prueba final de MCF 202.}
\CommentTok{\#cuantitativo}
\CommentTok{\#Numero de hermanos}
\CommentTok{\#cuantitativos}

\CommentTok{\#Problema 2:}
\CommentTok{\#Elija un objeto (cualquier objeto, por ejemplo, animales, plantas, países, instituciones, etc.) y obtenga}
\CommentTok{\#una lista de 4 variables: 2 cuantitativas y 2 categóricas}

\CommentTok{\#num \textless{}{-} c(1:2)}
\CommentTok{\#Cuantitativa \textless{}{-} c("Edades", "Numeros")}
\CommentTok{\#categoricas \textless{}{-} c("Fromas de pago", "Tipos de materiales")}

\CommentTok{\#Tabla \textless{}{-} data.frame(num, Cuantitativa, categoricas)}

\CommentTok{\#Problema 3:}

\CommentTok{\#Considere una variable con valores numéricos que describen formas electrónicas de expresar opiniones}
\CommentTok{\#personales: 1 = Twitter; 2 = correo electrónico; 3 = mensaje de texto; 4 = Facebook; 5 = blog. ¿Es}
\CommentTok{\#esta una variable cuantitativa o cualitativa? Explique.}

\CommentTok{\#1\_Twitter}
\CommentTok{\#cuantitativa, ya que se pueden contar los usuarios activos y sacar una media a partir de estos}

\CommentTok{\#2\_correos electronicos}
\CommentTok{\#cuantitativa, ya que se puede obtener los correos que se envian en un lapzao de x horas}

\CommentTok{\#3\_mensajes de texto}
\CommentTok{\#cauntitativa, poque se puede sacar el numero de mensajes enviados al dia por persona }

\CommentTok{\#4\_facebook}
\CommentTok{\#cuantitativa, se puede obtener el numero de personas activas simultaniamente, las reacciones por foto,etc.}

\CommentTok{\#5\_Blog}
\CommentTok{\#cuantitativas, se puede obtener cuantos blogs hace una persona po dia, los blogs ceraos en un dia por todos los usuarios.}

\CommentTok{\#Problema 4:}

  \CommentTok{\#Para cada pregunta de investigación, (1) identifique a los individuos de interés (el grupo o grupos que                                                                           se están estudiando), (2) identifique la (s) variable (s) (la característica sobre la que recopilaríamos                                                                                                                                               datos) y (3) determine si cada variable es categórico o cuantitativo.}

\CommentTok{\#¿Cuál es la cantidad promedio de horas que los estudiantes de universidades públicas trabajan}
\CommentTok{\#cada semana? }

\CommentTok{\#ht \textless{}{-} c(10, 14, 12, 18, 23, 15, 6, 9, 14, 24)}

\CommentTok{\#mean(ht)}

 \CommentTok{\# ¿Qué proporción de todos los estudiantes universitarios de México están inscritos en una}
\CommentTok{\#universidad pública?}

\CommentTok{\#EdU \textless{}{-} (230000)}
\CommentTok{\#EdUPrivadas \textless{}{-} (170000)}
\CommentTok{\#EdU {-} EdUPrivadas}


 \CommentTok{\# En los universidades públicas, ¿las estudiantes femeninas tienen un promedio de CENEVAL}
\CommentTok{\#más alto que los estudiantes varones? si}

  \CommentTok{\#¿Es más probable que los atletas universitarios reciban asesoramiento académico que los atletas}
\CommentTok{\#no universitarios? no}

 \CommentTok{\# Si reuniéramos datos para responder a las preguntas de la investigación anterior, ¿qué datos}
\CommentTok{\#podrían analizarse mediante un histograma? ¿Cómo lo sabes?.}
\CommentTok{\#si, porque mediante el histograma puedes ver el numero de variables que estas tomando para responder las preguntas}
\end{Highlighting}
\end{Shaded}


\end{document}
